\chapter{Conclusion}\label{chap-concl}

\section{Summary}\label{sec-sum}
Ozdal and Wong's program is an automatic single-layer bus routing program that matches wire lengths on high-speed boards. We try to improve the program so it becomes more general to accommodate different input nets. The program now does not have to route inside a uniform grid. The user can specify left and right boundaries to model the obstacles that might be encountered during PCB routing. Each individual net's boundary does not have to be monotonic anymore. Most of the cells inside the grid's boundary are available for the current net as long as they are not essential for the remaining nets to connect the two terminal pins. Therefore, the earlier nets can more likely finish routing than the later nets. The nets that have finished routing are drawn in the solution grid right away. As a result, the half-finished routing result is still available for the user when a net fails to satisfy the min-length or max-length bound.

During the actual routing for the net, the program makes an effort to have less bending in the path by giving priority to the previous direction. However, those routes can potentially waste cells, so the program gives the user the capability to enforce a cap on the directions that waste cells for the remaining nets. The program makes more cells available by making the last row of the grid routable for each net. As a result, more nets can meet the minimum-length constraint with tighter boundary.

The program can also use the diagonal wires in cases where the maximum-length bound cannot be satisfied by the minimum-path length. A second program, \textit{diagonal}, is used to preprocess the input file before the \textit{route} program. It checks the minimum-path distance for each net against the maximum-length bound and routes part of the grid in diagonal wire in case the min-path distance is less than the max-length bound. The diagonal region is chosen to be either at the bottom, at the top, or in the middle, depending on which region can contain the diagonal wires. The diagonal wire length has to be uniform for every wire to avoid the challenge of routing horizontal, vertical, and diagonal wires in the same region. After all the nets finish routing, the length of each net is automatically displayed in the terminal for the user.

With all these modifications, the single-layer bus routing program can be used without the original limitations of the nets input and environment.

\section{Future Work}\label{sec-fut}
There are some improvements that can be made to this program. The \textit{diagonal} program assumes that all nets use diagonal wires in the same direction. In reality, the nets can be in two different directions, $45^\circ$ left and $45^\circ$ right. As a result, they will need the program to draw the diagonal wires in these two directions in the same diagonal region. Furthermore, there are cases where it is impossible to satisfy the maximum-length bound for all nets using diagonal wire of uniform length.
Therefore, different diagonal wire lengths for different requirements can potentially solve this problem. However, the problem of routing wires of different lengths can be challenging. In addition, making the first row routable by the nets provides even more grid cells during routing. Lastly, the program should make a more intelligent guess of which mid-point to choose during routing, rather than taking a trial-and-error approach, which wastes a lot of runtime.
